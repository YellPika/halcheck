\section{Summary}

\begin{frame}{\halcheck{} --- Summary}
  \begin{itemize}
    \item Every generator in \halcheck{} is built from a small set of first-order core functions, resulting in a ``direct-style'' interface.
    \item Every core function can be overridden.
    \begin{itemize}
      \item Enables custom generation strategies (e.g., random vs exhaustive).
      \item Enables local modifications to generator behaviour (e.g., disabling shrinking).
    \end{itemize}
    \item Users must sometimes annotate functions with \cppinline{gen::group} to get proper shrinking behaviour.
  \end{itemize}
\end{frame}

\begin{frame}{\halcheck{} --- In Progress}
  \begin{columns}[T,onlytextwidth]
    \column{0.5\textwidth}
    \begin{block}{New strategies:}
      \begin{itemize}
        \item \cppinline{ordered} (SmallCheck/LeanCheck)
        \item Coverage-guided (fuzztest)
              \begin{itemize}
                \item Requires support for \alert{mutations}.
              \end{itemize}
        \item Learning-based (RLCheck)
        \item Reproducing test-cases
      \end{itemize}
    \end{block}

    \column{0.5\textwidth}
    \begin{block}{Test framework integration:}
      \begin{itemize}
        \item Google Test
        \item CUnit
        \item doctest (partially done)
      \end{itemize}
    \end{block}
  \end{columns}
\end{frame}
